\documentclass[11pt]{article}

% Change "review" to "final" to generate the final (sometimes called camera-ready) version.
% Change to "preprint" to generate a non-anonymous version with page numbers.
\usepackage[]{acl}
\usepackage{times}
\usepackage{latexsym}
\usepackage{comment}

% For proper rendering and hyphenation of words containing Latin characters (including in bib files)
\usepackage[T1]{fontenc}
% For Vietnamese characters
% \usepackage[T5]{fontenc}
% See https://www.latex-project.org/help/documentation/encguide.pdf for other character sets

% This assumes your files are encoded as UTF8
\usepackage[utf8]{inputenc}
\usepackage{microtype}
\usepackage{inconsolata}
\usepackage{graphicx}

% If the title and author information does not fit in the area allocated, uncomment the following
%
%\setlength\titlebox{<dim>}
%
% and set <dim> to something 5cm or larger.

\title{Group 65 Progress Report:\\Asteroid Classification using Machine Learning}


\author{Claire Nielsen, Jake Read, Rebecca Di Filippo \\
  \texttt{\{nielsc2, readj9, diflilr\}@mcmaster.ca} }

%\author{
%  \textbf{First Author\textsuperscript{1}},
%  \textbf{Second Author\textsuperscript{1,2}},
%  \textbf{Third T. Author\textsuperscript{1}},
%  \textbf{Fourth Author\textsuperscript{1}},
%\\
%  \textbf{Fifth Author\textsuperscript{1,2}},
%  \textbf{Sixth Author\textsuperscript{1}},
%  \textbf{Seventh Author\textsuperscript{1}},
%  \textbf{Eighth Author \textsuperscript{1,2,3,4}},
%\\
%  \textbf{Ninth Author\textsuperscript{1}},
%  \textbf{Tenth Author\textsuperscript{1}},
%  \textbf{Eleventh E. Author\textsuperscript{1,2,3,4,5}},
%  \textbf{Twelfth Author\textsuperscript{1}},
%\\
%  \textbf{Thirteenth Author\textsuperscript{3}},
%  \textbf{Fourteenth F. Author\textsuperscript{2,4}},
%  \textbf{Fifteenth Author\textsuperscript{1}},
%  \textbf{Sixteenth Author\textsuperscript{1}},
%\\
%  \textbf{Seventeenth S. Author\textsuperscript{4,5}},
%  \textbf{Eighteenth Author\textsuperscript{3,4}},
%  \textbf{Nineteenth N. Author\textsuperscript{2,5}},
%  \textbf{Twentieth Author\textsuperscript{1}}
%\\
%\\
%  \textsuperscript{1}Affiliation 1,
%  \textsuperscript{2}Affiliation 2,
%  \textsuperscript{3}Affiliation 3,
%  \textsuperscript{4}Affiliation 4,
%  \textsuperscript{5}Affiliation 5
%\\
%  \small{
%    \textbf{Correspondence:} \href{mailto:email@domain}{email@domain}
%  }
%}

\begin{document}
\maketitle
% \begin{abstract}
% \end{abstract}

\section{Introduction}

\textit{Here, write a brief introduction to the problem you are solving. This can be adapted from your problem description and motivation from the original proposal. This should be around 0.25-0.5 pages. }\newline

\noindent There are tens of thousands of Near Earth Asteroids (NEAs) on orbit paths close to Earth, with more being discovered daily \citep{NASANEAs}. Some of these become classified as Potentially Hazardous Asteroids (PHAs), as they become large enough to potentially cause a problem. As more and more NEAs are discovered every day, it would be helpful to develop a machine learning classification system that would be able to classify NEAs to determine which could be hazardous. Classification of asteroids requires time-consuming analysis of a variety of features. An autonomous model would allow space agencies to dedicate more resources to observation of the detected PHAs and developing mitigation strategies. \newline

\noindent Accurate and expedient classification of asteroids is crucial to determining potential risks to planetary health. This project will develop a model to classify NEAs based on physical and orbital characteristics to determine which could be hazardous to Earth. This project will clean and pre-process the dataset, selects relevant features, and trains a model using machine learning to classify asteroids as hazardous or non-hazardous. We will handle class imbalance using techniques like oversampling( e.g., SMOTE)), or weighted loss, and then evaluate the model. \newline

\noindent The dataset we will use contains an existing classification for, as well as features of thousands of discovered asteroids. The dataset is derived from NASA and JPL's Small-body and Asteroid database. The dataset contains 45 features of any known object, and we will engineer a feature set that yields the best results, clean the data and handle imbalance. \newline

\section{Related Work}

\textit{Here, talk about the related work you encountered for your approach. Cite at least 5 references. Refer to item 2. No one has done exactly your task? Write about the most similar thing you can find. This should be around 0.25-0.5 pages.} \newline 

\noindent \citet{NASANEOs} defines Potentially Hazardous Asteroids (PHAs) based on parameters that affect the asteroid's potential to be threatening, including its closest distance to Earth, size and albedo. There have been several systems developed using various machine learning models. \newline

\noindent Results of a comparison of machine learning models to classify NEAs were published in 2022, and included comparison of logistic regression, naive Bayes, support vector machines (SVMs), gradient boosting, and MultiLayer Perceptrons (MLPs). The article states that \textit{multilayer perception} and \textit{gradient boosting} yielded the most accurate results \citep{CompareML}. \newline

\noindent A further indicator of the performance of gradient boosting came from a classification project using the NGBoost classifier, a gradient boosting framework that uses natural gradient descent for optimization. Using NGBoost produced an overall accuracy of 99.22\% \citep{NGBoost}. \newline

\noindent Another article published in 2019 compares the Hierarchal Clustering Method (HCM) to a newer, \textit{supervised learning HCM} method. The supervised HCM method proved superior to the classical method, correctly identifying all asteroids within the target family and and yielding an accuracy of about 85\% \citep{HCM}. \newline

\noindent A different approach was taken by \citet{SVM}, who elected to observe subgroups of NEAs with high concentrations of PHAs, to better determine which characteristics can be used as flags of hazardous asteroids. Using a \textit{Support Vector Machine (SVM)} model, the extracted subgroups of NEAs contained about 90\% of the real and virtual PHAs. \newline 

\section{Dataset}

\textit{You should write about your dataset here, following the guidelines regarding item 1. This section may be 0.5-1 pages. Depending on your specific dataset, you may want to include subsections for the preprocessing, annotation, etc.} \newline

\noindent We are using a public dataset available on Kaggle, licensed under Open Data Commons Open Database License, by user sakhawak18. The database is called \textit{Asteroid Dataset}, found \href{https://www.kaggle.com/datasets/sakhawat18/asteroid-dataset}{here} \cite{10.1007/978-981-19-7528-8_4}. As stated in the summary, it is officially maintained and updated weekly, and has a Kaggle-calculated usability score of 10.00. The original source of the data is NASA's Jet Propulsion Laboratory Small-Body and Asteroid databases, containing both orbital and physical properties for hundreds of thousands of known asteroids \citep{database}. \newline

\noindent The dataset contains orbital and physical properties of thousands of discovered and analyzed asteroids. There is a target label \texttt{pha} present, representing the classification to be predicted by the model. These are the features our model will use, and are as follows:

\begin{itemize}
    \item \textbf{Flags:} \texttt{neo} (Near-Earth Object flag), \texttt{pha} (Potentially Hazardous Asteroid flag, 1 = hazardous, 0 = non-hazardous)

    \item \textbf{Physical Properties:} \texttt{H} (absolute magnitude), \texttt{diameter} (km), \texttt{albedo} (surface reflectivity), \texttt{diameter\_sigma} (uncertainty in diameter)

    \item \textbf{Orbital Elements:}
    \begin{itemize}
        \item Core orbital parameters: \texttt{e} (eccentricity), \texttt{a} (semi-major axis), \texttt{q} (perihelion distance), \texttt{i} (inclination), \texttt{om} ($\Omega$, longitude of ascending node), \texttt{w} ($\omega$, argument of perihelion), \texttt{ma} (mean anomaly)
        \item Derived orbital distances: \texttt{ad} (aphelion distance), \texttt{moid} (minimum orbit intersection distance), \texttt{moid\_ld} (MOID in lunar distances)
        \item Motion and timing: \texttt{n} (mean motion), \texttt{tp} (time of perihelion passage), \texttt{per} (orbital period in days), \texttt{per\_y} (orbital period in years)
    \end{itemize}

    \item \textbf{Uncertainties (Standard Deviations):} \texttt{sigma\_e}, \texttt{sigma\_a}, \texttt{sigma\_q}, \texttt{sigma\_i}, \texttt{sigma\_om}, \texttt{sigma\_w}, \texttt{sigma\_ma}, \texttt{sigma\_ad}, \texttt{sigma\_n}, \texttt{sigma\_tp}, \texttt{sigma\_per}

    \item \textbf{Model Fit Quality:} \texttt{rms} (root-mean-square residual, indicating fit accuracy of the orbital solution)

    \item \textbf{Target Label:} \texttt{pha} (Potentially Hazardous Asteroid flag — the classification label to be predicted)
\end{itemize}

The dataset also contains identifier, metadata, epoch and reference data features that will not be used in our model. The excluded features are: 

\begin{itemize}
    \item \textbf{Identifiers and Metadata:} \texttt{id}, \texttt{spkid}, \texttt{full\_name}, \texttt{pdes}, \texttt{name}, \texttt{prefix}, \texttt{orbit\_id}, \texttt{class}

    \item \textbf{Epoch and Reference Data:} \texttt{epoch}, \texttt{epoch\_mjd} (Modified Julian Date), \texttt{epoch\_cal} (calendar format), \texttt{equinox}, \texttt{tp\_cal} (perihelion passage in calendar format) \newline 
\end{itemize}

\noindent We will import the dataset using the Kaggle API, complying with its terms of service. Once the dataset is imported it will then be preprocessed, which includes a number of steps. The dataset contains some non-numeric columns in the identifiers and metadata (including but not limited to \texttt{id}, \texttt{name}, and \texttt{class}) that will be dropped as they cannot contribute to prediction. Some categorical categories, including the \texttt{pha} flag, will be converted to numerical binary values and processed as such. Any string values remaining in the dataset will be forced to \texttt{NaN} to ensure that no unpredicted errors occur in processing. The resulting missing values will then be handled by converting them to the column mean. \newline

\section{Features}
\label{sec:Features}

\textit{Describe any features you used for your model, or how your data was input to your model. Are you doing feature engineering or feature selection? Are you learning embeddings? Is it all part of one neural network? Refer to item 2. This may range from 0.25 pages to 0.5 pages.} \newline

\noindent As mentioned, the model handles most but not all features provided by the dataset. This is because the dataset provides data that is not required for predictive purposes, which in this case includes identifiers, metadata, epoch and reference data. The included features contain the data that is most relevant to the categorization of an NEA as hazardous. \newline

\noindent NASA categorizes PHAs based on the value of \texttt{moid} (minimum orbit intersection distance), \texttt{H} (absolute magnitude), and \texttt{albedo} (surface reflectivity). The minimum orbit intersection distance is the closest the asteroid will come to Earth on its natural orbit path, with hazardous asteroids having a \texttt{moid} $\leq 0.05$au, or astronomical units (149,597,870,700 m, which approximates the distance between the Earth and the sun). The absolute magnitude of an asteroid represents the visual magnitude an observer would record if the asteroid were placed exactly 1au away and 1au from the Sun at a zero phase angle (effectively measuring size). PHAs are characterized by an absolute magnitude \texttt{H} $\leq 22$. This magnitude translates to asteroids that are no more than 140m in diameter, and corresponds to an assumed \texttt{albedo} = 0.14. The albedo of an asteroid represents its ratio of the light received to light reflected by that body, ranging from 0 (pitch black) to 1 (perfect reflector). Our model will take in these features as well as the others to produce an accurate prediction of the \texttt{pha} flag. \newline

\noindent We are using feature selection in our model. Feature engineering involves creating new features based on overlapping/redundant features from the original dataset. We may have some very similar features, but at this point have chosen not to engineer them for use with this model. We will not be using embedding, as any relevant data to the prediction is represented numerically, and we do not have to process string data or images. 

\section{Implementation}

Describe your model and implementation here. Refer to item 4. This may take around a page.

\section{Results and Evaluation}

How are you evaluating your model? What results do you have so far? What are your baselines? Refer to item 5. This may take around 0.5 pages.

\section{Feedback and Plans}

Write about your plans for the remainder of the project. This should include a discussion of the feedback you received from your TA, and how you plan to improve your approach. Reflect on your implementation and areas for improvement. Refer to item 6. This may be around 0.5 pages.

\begin{comment}
\section{Template Notes}

You can remove this section or comment it out, as it only contains instructions for how to use this template. You may use subsections in your document as you find appropriate.

\subsection{Tables and figures}

See Table~\ref{citation-guide} for an example of a table and its caption.
See Figure~\ref{fig:experiments} for an example of a figure and its caption.


\begin{figure}[t]
  \includegraphics[width=\columnwidth]{example-image-golden}
  \caption{A figure with a caption that runs for more than one line.
    Example image is usually available through the \texttt{mwe} package
    without even mentioning it in the preamble.}
  \label{fig:experiments}
\end{figure}

\begin{figure*}[t]
  \includegraphics[width=0.48\linewidth]{example-image-a} \hfill
  \includegraphics[width=0.48\linewidth]{example-image-b}
  \caption {A minimal working example to demonstrate how to place
    two images side-by-side.}
\end{figure*}

\subsection{Citations}

\begin{table*}
  \centering
  \begin{tabular}{lll}
    \hline
    \textbf{Output}           & \textbf{natbib command} & \textbf{ACL only command} \\
    \hline
    \citep{Gusfield:97}       & \verb|\citep|           &                           \\
    \citealp{Gusfield:97}     & \verb|\citealp|         &                           \\
    \citet{Gusfield:97}       & \verb|\citet|           &                           \\
    \citeyearpar{Gusfield:97} & \verb|\citeyearpar|     &                           \\
    \citeposs{Gusfield:97}    &                         & \verb|\citeposs|          \\
    \hline
  \end{tabular}
  \caption{\label{citation-guide}
    Citation commands supported by the style file.
  }
\end{table*}

Table~\ref{citation-guide} shows the syntax supported by the style files.
We encourage you to use the natbib styles.
You can use the command \verb|\citet| (cite in text) to get ``author (year)'' citations, like this citation to a paper by \citet{Gusfield:97}.
You can use the command \verb|\citep| (cite in parentheses) to get ``(author, year)'' citations \citep{Gusfield:97}.
You can use the command \verb|\citealp| (alternative cite without parentheses) to get ``author, year'' citations, which is useful for using citations within parentheses (e.g. \citealp{Gusfield:97}).

\subsection{References}

\nocite{Ando2005,andrew2007scalable,rasooli-tetrault-2015}

Many websites where you can find academic papers also allow you to export a bib file for citation or bib formatted entry. Copy this into the \texttt{custom.bib} and you will be able to cite the paper in the \LaTeX{}. You can remove the example entries.

\subsection{Equations}

An example equation is shown below:
\begin{equation}
  \label{eq:example}
  A = \pi r^2
\end{equation}

Labels for equation numbers, sections, subsections, figures and tables
are all defined with the \verb|\label{label}| command and cross references
to them are made with the \verb|\ref{label}| command.
This an example cross-reference to Equation~\ref{eq:example}. You can also write equations inline, like this: $A=\pi r^2$.

\end{comment}
% \section*{Limitations}

\section*{Team Contributions}

\textit{Write in this section a few sentences describing the contributions of each team member. What did each member work on? Refer to item 7.} \newline

\noindent \textbf{Claire Nielsen} handled the research into related works and strategies, and wrote the Progress Report. \newline

\noindent \textbf{Rebecca Di Filippo} and \textbf{Jake Read} handled the coding of the model up to this point in the project. Rebecca handled the preprocessing of the dataset, feature engineering and selection, and baseline model for comparison. Jake wrote the starting gradient boosting algorithm, loss function, and optimization technique. Rebecca and Jake collaborated on the evaluation strategies for the model (including the training/testing split, cross-validation, and metrics). 

% Bibliography entries for the entire Anthology, followed by custom entries
%\bibliography{custom,anthology-overleaf-1,anthology-overleaf-2}

% Custom bibliography entries only
\bibliography{custom}

% \appendix

% \section{Example Appendix}
% \label{sec:appendix}

% This is an appendix.

\end{document}
